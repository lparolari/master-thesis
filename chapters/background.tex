%!TEX root = ../dissertation.tex

\begin{savequote}[75mm]
Nulla facilisi. In vel sem. Morbi id urna in diam dignissim feugiat. Proin molestie tortor eu velit. Aliquam erat volutpat. Nullam ultrices, diam tempus vulputate egestas, eros pede varius leo.
\qauthor{Quoteauthor Lastname}
\end{savequote}

\chapter{Background}

TODO

* object detector (theory)
* rnn + lstm
* textual embedding

\section{Object detection and recognition}

Object detection (OD) and object recognition (OR) systems are crucial
in a wide variety of everyday tasks such us face detection, image and
video databases information retrieval, surveillance applications,
driver assistance, self drive, robotics, automation and, specially in
vision tasks, it is a foundamental building block used to extract
information from images.

The primary essence of those systems can be broken down into two
parts: to locate objects in a scene such us by drawing a bounding box
around the object (object detection) and later to classify the objects
based on the classes it was trained on (obejct recognition). OD and OR
are Often used together and for this reason the name of two tasks is
usually used interchangeably.

We can group the state-of-the-art detection systems in two main
approaches: one-stage methods (YOLO - You Only Look Once \todo{CITE:
yolo}, SSD - Single Shot Detection \todo{CITE: ssd}) and two-stage
approaches (R-CNN, Fast R-CNN, Faster R-CNN \todo{CITE: RCNN}).

\newthought{One stage methods}

\begin{enumerate}[label=(\alph*)]
  \item YOLO -- \emph{You Only Look Once} is a state-of-the-art,
  real-time object detection system which offers extreme levels of
  speed and accuracy. YOLO reframes object detection as a regression
  problem to spatially separated bounding boxes and associated class
  probabilities. A single neural network predicts bounding boxes and
  class probabilities directly from full images in one evaluation.
  Since the whole detection pipeline is a single network, it can be
  optimized end-to-end directly on detection performance. This
  architecture offers some advantages wrt to other methods: first,
  YOLO is extremely fast and second, it reasons globally about the
  image when making prediction implicitly encoding contextual
  information about classes as well as their appearance. Last but not
  least, YOLO is more likely to learn generalizable representations of
  objects because of its global reasoning on image.
  \item SSD -- \emph{Single Shot Detection} another state-of-the-art
  method for detecting objects in images using a single deep neural
  network. SSD discretizes the output space of bounding boxes into a
  set of default boxes over different aspect ratios and scales per
  feature map location and, at prediction time, the network generates
  scores for the presence of each object category in each default box
  and produces adjustments to the box to better match the object
  shape. Additionally, the network combines predictions from multiple
  feature maps with different resolutions to naturally handle objects
  of various sizes. Also, SSD offers high speed performance due its
  single network and for this reason it is easy to train and
  straightforward to integrate into systems that require a detection
  component. In terms of accuracy, experimental results confirm that
  SSD has competitive accuracy also in low resolution compared to
  methods that utilize an additional object proposal step.
\end{enumerate}

\newthought{Two stage methods}

\begin{enumerate}[label=(\alph*)]
  \item R-CNN -- \emph{Regions with CNN} is the first performant
  object detection system ever built. It is composed by a two-stage
  architecture: in the first step the Selective Search algorithm
  generates around 2000 category-independent region proposals for the
  input image, in the last step instead it extracts a fixed-length
  feature vector from each proposal using a CNN (hence the name
  R-CNN), and then classifies each region with category-specific
  linear SVMs. The method shows interesting results and can be applied
  also with scarce data availability: the system can be pretrained and
  the fine-tuned. In temrs of computation time, R-CNN performs some
  expensive operations required for greedy non-maximum suppression,
  amogn others, but on relatively small inputs.
  \item Fast R-CNN -- The Fast R-CNN model was built to counter a few
  drawbacks of the previous R-CNN model. In this approach, similar to
  the previous, Selective Search is used to generate region proposals
  but the input image is fed to a CNN and a convolutional feature map
  is generated from it which is then used to identify the regions and
  combine them into larger squares by using a RoI pooling layer. A
  softmax layer is finally used to predict the class of the proposed
  region.
  \item Unlike R-CNN and Fast R-CNN, Faster R-CNN does not use
  Selective Search which is a slow process. Instead, it allows the
  network to learn the region proposals throught a separate network,
  able to predict the region proposals. The predicted proposals are
  then pooled into larger squares using the RoI pooling layer which is
  then finally used to classify the image.
\end{enumerate}

\todo{ADD: quale abbiamo scelto? attenzione a quanto già detto in introduzione (potrebbe essere spostato qui...)}
