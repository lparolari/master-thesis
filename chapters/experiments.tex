%!TEX root = ../dissertation.tex
\begin{savequote}[75mm]
Nulla facilisi. In vel sem. Morbi id urna in diam dignissim feugiat. Proin molestie tortor eu velit. Aliquam erat volutpat. Nullam ultrices, diam tempus vulputate egestas, eros pede varius leo.
\qauthor{Quoteauthor Lastname}
\end{savequote}

\chapter{Experiment and results}

SCALETTA

* datasets
  * flickr
  * referit
* evaluation metric
* impl details
* model selection
* results
* qualitative results

\section{Datasets}

In this section we describe Flickr30K and ReferIt, the two datasets
used for the experimental assessment. \todo{??? Furthermore, we
describe two other noteworthy and popular datasets available in
literature}

\subsection{Flickr30K}

The Flickr30K Entities \todo{CITE: Flickr30K Entities} is a dataset
built on top of Flickr30K \todo{CITE: From image descriptions to
visual denotations: New similarity metrics for semantic inference over
event descriptions}. It is very popular and it's usually referred as Flickr30K or, simply, as Flickr. Compared to the original Flickr30K dataset, Flickr30K Entities focus on the task of grounding textual mentions of entities in image and augment data with some annotations.

Annotations consists of cross-caption coreference chains linking
mentions of the same entities together with bounding boxes localizing
those entities in the image, and are collected throught a
crowdsourcing protocol. These annotations are highly structured and
vary in complexity from image to image, since images vary in the
numbers of clearly distinguishable entities they contain, and
sentences vary in the extent of their detail. Further, there are
ambiguities involved in identifying whether two mentions refer to the
same entity or set of entities, how many boxes (if any) these entities
require, and whether these boxes are of sufficiently high quality. To
tackle this problem, compounded by the unreliability of crowdsourced
judgments, they administer a pipeline of simpler and atomic tasks
which can be grouped into two main stages: coreference resolution, or
forming coreference chains that refer to the same entities, and
bounding box annotation for the resulting chains. This workflow
provides two advantages: first, identifying coreferent mentions helps
reduce redundancy and save box-drawing effort; and second, coreference
annotation is intrinsically valuable.

The coreference resolution problem is solved by first chunking
information given in Flickr30K captions to identify potential entity
mentions. Each chunk, i.e., noun phrase (NP), is a potential entity
mention. For example, the frase ``A man in an orange hat'' is chunked
in two noun phrases, namely ``A man'' and ``an orange hat'', which are
the two mentions. Given $M$ a document containing noun phrases
originated from captions of an image, a worker is required to specify
whether two given mentions $m$ and $m'$ refer to the same entity. If
the answer is positive, a link between the two mentions is addded,
creating a coreference chain. Tipically, this would require $O(|M|^2)$
which is the cost of all pairwise links. But since $M$ usually
contains multiple mentions that refer to the same set of entities, the
number of coreference chains is bounded by a number much smaller than
$|M|$, on average. Also, the bound is lowered by making two assumptio.
First, they assume that mentions from the same captions cannot be
coreferent; second, they categorize each mention into eight
coarse-grained types using manually constructed dictionaries (people,
body parts, animals, clothing/color, instruments, vehicles, scene, and
other), and assume mentions belonging to different categories cannot
be coreferent. The goodnees of the strategy is empirically shown by a
small-scale test on 200 images and the coreference chains are verified
by another task which asks to a worker whether given mentions in
coreference chain refer to the same entity. Their results shows that
only $9.8\%$ of all coreference chains are not correct.

Bounding box annotation is carried out with a workflow composed by
four tasks administered to workers, namely: box requirement, box
drawing, box quality and box coverage. In box requirement, a worker
needs to specify, whether at least one bounding box can be drawn in
image for given mention. If the response is negative, the mention
leaves the workflow, otherwise it goes throught the box drawing stage.
Here, a worker should draw a box as tight as possible around the
mentioned entity. The main source of difficulty in this stage is due
to mentions that refer to multiple entities. If a box is drawn, then
the mention-box pair proceeds to box quality stage. At this point,
drawn box is evaluated in terms of redundancy (are there any box that
already covers the same entity?), accuracy (is the box tight around
the entity?), distinctiveness (can a box be drawn for single entity
instead of covering multiple elements?). If the answer is positive,
the example goes to the last stage. In box coverage, workers decide
whether all required boxes are present in image. 

During the annotation process, workers are subjected to quality
control process. Workers must pass a test before being allowed to
annotate dataset examples. Also, during the annotation process,
workers are asked some verification questions (question with known
answer, written by authors).

Above work leads to Flicker30K Entities, taht contains 32K images,
159K sentences, 275K bounding boxes, and 360K noun phrases where each
image is associated with five sentences with a variable number of noun
phrases, and each noun phrase is associated with a set of bounding
boxes ground truth coordinates. 
