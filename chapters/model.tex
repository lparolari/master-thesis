%!TEX root = ../dissertation.tex
\begin{savequote}[75mm]
This is some random quote to start off the chapter.
\qauthor{Firstname lastname}
\end{savequote}

\chapter{Model and approach}
\label{ch:model}

\todo{rivedere il titolo del capitolo}

\section{Model}

In the following sections we describe the model architecture, outlined
in Fig.\ref{fig:model-architecture}. We first delineate the visual and
textual branch in Sec.~\ref{subsec:visual-branch} and
\ref{subsec:textual-branch}. We then present the concept similarity,
our novel contribution, in Sec.~\ref{subsec:similarity-branch}, along
with the prediction and loss modules in
Sec.~\ref{subsec:prediction-module} and \ref{sec:loss}. We then
conclude highlighting the training and inference protocol in
Sec.~\ref{sec:training-and-inference}.

TODO \todo{CAMBIARE IMMAGINE, QUESTA é QUELLA SBAGLIATA}

\begin{figure}
  \centering
  \includegraphics[width=1.0\textwidth]{figures/model-architecture.png}
  \caption[Model architecture overview]{ Our model architecture
    overview. Initially, the image is processed by a pre-trained
    Faster R-CNN object detector in order to extract all the proposals
    bounding boxes from which the spatial features are generated.
    Then, the model generates textual features from the input noun
    phrase by first retrieving each word embedding and then using an
    LSTM network. Finally, the model fuses together all the visual,
    spatial, and textual features by the Fusion Operator. Those
    features are then used by the model in combination with textual
    feature to predict a score of similarity between the two. The
    concept branch predict another similarity score between the word
    embedding of concept the phrase and the class label on proposal.
    Such score is applied to models predictions as a prior and fed
    into the loss module that optimizes positive scores to be larger
    than negative scores. }
  \label{fig:model-architecture}
\end{figure}

\subsection{Visual Branch}
\label{subsec:visual-branch}

Given an image $\bm{I}$, we extract a set of $k$ bounding box
proposals $\calP_{\bm{I}} = \{ \bm{p}_i \}^k_{i=1}$ by means of a
pre-trained object detector where $p_i \in \Rset^4$, jointly with
features $H^v = \{ \bm{h}^v_i \}^k_{i=1}$, where $\bm{h}^v_i \in
\Rset^v$. The features represent the internal object detector
activation values before the classification layers and regression
layer for bounding boxes. Moreover, our model extracts the spatial
features $H^s = \{ \bm{h}^s_i \}^k_{i=1}$, where $\bm{h}^s_i \in
\Rset^s$ from all the bounding boxes proposals, with the spatial
features for the proposal $\bm{p}_i$ defined as:
\begin{equation}
  \bm{h}^s_i = \left[ \frac{x1}{wt}, \frac{y1}{ht}, \frac{x2}{wt}, \frac{y2}{ht}, \frac{(x2 - x1) \times (y2 - y1)}{wt \times ht}  \right]
\end{equation}
where $(x1, y1)$ refers to the top-left bounding box corner, $(x2,
y2)$ refers to the bottom-right bounding box corner, $wt$ and $ht$ are
the width and height of the image, respectively. Both visual and
spatial features are then concatenated and projected, thus leading to
a set of new vectorial representations $H^{||} = \{ \bm{h}^{||}_{jz}
\}_{j \in [1, \ldots, m], z \in [1, \ldots, k]}$, where vectors
$\bm{h}^{||}_{jz}$ are defined as:
\begin{equation}
  \bm{h}^{||}_{jz} = \bm{W}^{||} \left( \bm{h}^s_z || L1(\bm{h}^v_z) \right) + \bm{b}^{||}
  \label{eq:h-par-jz}
\end{equation}
where $||$ indicates the concatenation operator, $\bm{h}^{||}_{jz} \in
\Rset^c$, $\bm{W}^{||} \in \Rset^{c \times (s + v)}$ is a matrix of
weights, $\bm{b}^{||} \in \Rset^c$ is a bias vector, and $L1$ is the
L$1$ normalization function.

We also assume that the object detector returns, for each $\bm{p}_i$,
a probability distribution $Pr_{Cls}(\bm{p}_i)$ over a set $Cls$ of
predefined classes, i.e. the probability for each class $\zeta \in
Cls$ that the content of the bounding box $\bm{p}_i$ belongs to
$\zeta$. This information is typically returned by most of the object
detectors, and it will be used in Sec.~\ref{subsec:similarity-branch} to define the concept similarity.

\subsection{Textual Branch}
\label{subsec:textual-branch}

Regarding the textual features extraction, given a noun phrase
$\bm{q}_j$, initially all its words $W^{\bm{q}_j} = \{ w^{\bm{q}_j}_i
\}^l_{i=1}$ are embedded in a set of vectors $E^{\bm{q}_j} =
\{e^{\bm{q}_j}_i \}^l_{i=1}$ where $e^{\bm{q}_j}_i \in \Rset^w$, where
$w$ is the size of the embedding. Then, our model applies a LSTM
neural network (Sec.~\ref{subsec:gated-rnn}) to generate from the
sequence of word embeddings only one new embedding $\bm{h}^*_j$ for
each phrase $\bm{q}_j$. This textual features extraction is defined
as:
\begin{equation}
  \bm{h}^*_j = L1(LSTM(E^{\bm{q}_j}))
  \label{eq:h-star}
\end{equation}
where $\bm{h}^*_j \in \Rset^t$ is the LSTM output of the last word in
the noun phrase $\bm{q}_j$, and $L1$ is the L$1$ normalization
function.

\subsection{Similarity Branch}
\label{subsec:similarity-branch}

Along with visual and textual features we also compute a similarity
score between noun phrases and bounding boxes: the concept similarity
\cite{wang2019phrase}. The aim of such score is to capture the
semantic similarity between the content of a proposal and the concept
expressed by a phrase. But, instead of learning a complex multimodal
mapping function between features, which is very difficult to realize
in weakly supervised settings, we directly combine information given
by object detector and phrases. More precisely, we exploit those
information by leveraging on pretrained word embeddings. A good set of
word embeddings is able to capture, above all, the semantic similarity
between words, hence, words with same meaning should have similar
representation (Sec.~\ref{sec:word-embeddings}). Based on this
assumption, the similarity score can be trivially expressed as a
distance measure between two vectors in a space. Thus, we define the
concept similarity as a distance measure between to two word
embeddings, i.e., concepts. The first word embedding is extracted in a
way that it can express the content of a proposal. This information is
made available by the object detector, which usually return a
probability distribution over a set of labels, for each proposal
(Sec.~\ref{subsec:visual-branch}). We can easily gather the label that
best express proposal's content by simply taking label looking for the
one with maximum probability. This label is just a word (or, in some
case a combination of words), that can be naturally embedded in the
word embeddings space. The word embedding that express phrase concept,
instead, can be a word from the phrase or a combination of words.

Formally, we define $\EPI = \{\ePI_i \}^{k}_{i=1}$ the set of labels
embeddings build on the set of proposal $\calP^{\bm{I}}$. Each
$\ePI_i$ is a feature vector representing the label predicted with
maximum probability by the object detector for the $i$-th proposal.
Then, given a noun phrase $\bm{q}_j$ along with $E^{\bm{q}_j}$, we
compute the concept similarity score for each proposal $\bm{p}_i$ as:
\begin{equation}
  \bm{S}^c_{ji} = \fsim \left( \xi_j, \ePI_i \right),
\end{equation}
where $\fsim$ is a similarity measure such us the cosine similarity,
and $\xi_j$ is the embedding representing the concept of the noun
phrase $\qj$. The new embedding $\xi_j$ can be computed with various
strategies, and we can group these into two main groups. The first is
the group where we can find operators that take into account the
entire phrase to generate a representative. A widely used strategy is
to generate a new embedding by averaging phrase's word embeddings:
\begin{equation}
  \xi_j = \frac{\sum^l_{i=1} \eqj_i}{| \Eqj |},
\end{equation}
however, there is an evidence that such averaging operations can
compromise the discriminativeness
\cite{wang2019phrase,datta2019align2ground}. In the second group we
find operators that consider a single word as a representative for the
phrase. A common strategy employed in literature is to select the last
word in the phrase \cite{wang2019phrase}. Specially in English
language, the last word of a noun phrase is usually the head of the
phrase. However, this strategy relies on strong assumptions on
language and how dataset is built.\footnote{In \cite{wang2019phrase},
they show how the \textit{last} strategy differs in performance on
different dataset: on Flickr30k Entities it is the second best
strategy while in ReferIt it doesn't even appear on the leaderboard.}
More complex strategies, instead, take into account external
information, such us the similarity wrt proposal class embeddings.
Here, we compute the similarity between word in noun phrase and
proposal class embeddings and then we select one word in the noun
phrase with maximum similarity wrt the detected concept in proposal:
\begin{equation}
  \xi_j = \argmax_{\eqj \in \Eqj} \{ \max g(\eqj) \},
  \label{eq:concept-embedding-max-sim}
\end{equation}
where
\begin{equation}
  g(\eqj) = \{ \fsim(\ePI, \eqj) \mid \ePI \in \EPI \}.
\end{equation}
The latter case is better explained by means of an example, such us
the one in Fig.~\ref{fig:concept-selection-example}. For the sake of
presentation, we assume to deal with an object detector trained on two
class labels $Cls = \{ \text{sky}, \text{animal} \}$. We are given an
image (in figure) and the query ``the blue elephant''. As described in
Eq.~\ref{eq:concept-embedding-max-sim}, our model computes all
similarities between the embedding of words in phrase $\eqj$ and the
embedding of proposal's class labels $\ePI$. The representative
concept for the query is defined as the embedding of the word with
maximum similarity wrt proposals. In our example, the word
``elephant'' has maximum similarity with the proposal labeled with
\textit{animal}, so it becomes the concept of the phrase. Scores
associated with this concept will be used at prediction stage
(Sec.~\ref{subsec:prediction-module}) to down- or up- weight
proposals.

\begin{figure}
  \centering
  \includegraphics[width=0.8\textwidth]{figures/concept-selection-example.png}
  \caption[Concept selection example]{ TODO }
  \label{fig:concept-selection-example}
\end{figure}

\subsection{Prediction Module}
\label{subsec:prediction-module}

Finally, the model predicts the probability $\bm{P}_{jz}$ that a given
noun phrase $\qj$ is referrred to a proposal bounding box $\bm{p}_z$
as:
\begin{equation}
  \bm{P}_{jz} = \fsim ( \bm{h}^{||}_{jz} , \bm{h}^*_j ).
\end{equation}

As noted in \cite{chen2018knowledge}, concept similarity is a direct
consequence of the intrinsic knwoledge convoyed by the object
detector. Such score can be useful to down-weight unrelated proposals.
Its effectiveness is due to the fact that the word embedding usually
captures the semantic similarity between concept in phrase and the
content of the bounding box, and thus, let the model focuses on
relevant proposals. Thus, a straightforward improvement to our base
model would be to include this knowledge and exploit it for making
predictions. Hence, predictions can be enhanced by multiplying the
two, treating $\bm{S}^c_{jz}$ as a weight matrix:
\begin{equation}
  \bm{\hat{P}}_{jz} = \bm{S}^c_{jz} * \bm{P}_{jz}.
\end{equation}
At first glance, this approach seems very promising because we force
the model to return predictions ``on steroid'' when there is a
semantic relation between phrase and proposal or downweighted when
little similarity is found. However, this relies on the assumption
that the embedding space and similarity measure we are using for words
perfectly captures the semantic similarity between them, and this is
not true. Also, we assume that proposals are classified with no error
by the object detector, neither this is true. Under those hypotesis,
such application of concept similarity cannot be fruitful. 

An example my clarify our statement. We are given $\bm{S}^c \in
\Rset^{1 \times 2}$, $\bm{S}^c = [0.1, 0.2]$ for an example composed by a
phrase and two proposal, and we know that the bounding box to ground
with our prhrase is the first. The only way the model has to output
the first proposal as the best bounding box, i.e., the bounding box to
ground, is to predict a score $p_{0}$ which is at least the double of
$p_{1}$, because
\begin{equation}
  \bm{\hat{P}} = [0.1 * p_{0}, 0.2 * p_{1}]
\end{equation}
and hence
\begin{equation}
\begin{split}
0.1 * p_{0} > 0.2 * p_{1} & \iff p_{0} >  \frac{0.2}{0.1} * p_{1} \\
  & \iff p_{0} > 2 * p_{1}.
\end{split}
\end{equation}
Here, $p_{j} = \bm{P}_{0,j}$. Thus, with such scores $\bm{S}^c$, when
the model predicts $0.1$ for $p_1$, it must learn to predict $> 0.2$
for $p_0$, when it predicts $0.5$ for $p_1$, it must learn to predict
$1.0$ (the maximum) for $p_0$ and when it predicts a value $> 0.5$ for
$p_1$ there is no way to predict a greater score for $p_0$. In
conclusion, the model isn't really able to learn and its completely at
the mercy of similarity scores.

A more convenient way of applying concept similarity to prediction is
instead to compute the mean between two. Eventually, we can also put a
weight $\lambda \in [0 .. 1]$ in order to balance
contributions. Formally:
\begin{equation}
  \bm{\hat{P}}_{jz} = (1 - \lambda) * \bm{S}^c_{jz} + \lambda * \bm{P}_{jz}.
\end{equation}
The major benefit of this approach is that model predictions are not
constrained to values defined by concept similarity: they co-work to
the final predictions.

\section{Loss Function}
\label{sec:loss}

In this section we present our loss function. Please note that in
weakly supervised settings, for a training example $\left( \bm{I}, S
\right)$ we are not given the query-proposal pair set $\{ ( q^{gt}_j,
p^{gt}_j ) \}^m_{j=1}$, where $m$ is the number of noun phrases.
Inspired by \cite{wang2020maf}, we adopt a contrastive loss. The
contrastive objective $\calL$ aims to learn the visual and textual
features by maximizing the similarity score between paired
image-caption elements and minimizing the score between the negative
samples (i.e., other irrelevant captions). Formally, given the image
$\bm{I}$, $Q_S = \{ \bm{q}_j \}^m_{j=1}$ the set of noun phrases built
on the sentence $S$ weakly linked to $\bm{I}$ and $Q'_S = \{ \bm{q}_h
\}^{m'}_{h=1}$ we define $\calL$ as:
\begin{equation}
  \calL = - \frac{1}{m} \sum^m_{j=1} \calL_j,
\end{equation}
where
\begin{equation}
  \calL_j = \log \frac{e^{\fmmsim(\bm{I}, \bm{q}_j)}}{\sum^{m'}_{h = 1} e^{\fmmsim'(\bm{I}, \bm{q}_{h})}}.
\end{equation}
Here, $\fmmsim$ and $\fmmsim'$ are the multimodal similarity functions
between image and positive/negative queries respectively. We define
\begin{equation}
  \fmmsim(\bm{I}, \bm{q}_j) = \max_{z} \bm{P}_{jz}, \qquad \text{and} \qquad \fmmsim'(\bm{I}, \bm{q}_h) = \max_z \bm{P}_{hi}
\end{equation}
where $\bm{P}_{jz}$ is the predicted similarity between query
$\bm{q}_j$ and proposal $\bm{p}_z$ with $\bm{p}_z \in \calP_{\bm{I}}$,
while $i = \argmax_z \bm{P}_{hz}$. Basically, the model optimizes the
similarity between each positive query $\bm{q}_j$ and the image
$\bm{I}$ to be higher, while the similarity between negative queries
$\bm{q}_h$ and image $\bm{I}$ to be smaller. Moreover, the similarity
function on negative queries $\fmmsim'$ bounds the model to consider
only bounding boxes affected by positive queries.

--- \todo{REMOVE BELOW}

We optimize the model to learn features representation in the
similarity space, such that the depiction of $\bm{p}_z$ and $\bm{q}_j$
are neighbors when $\bm{D}_{jz}$ is $1$, perpendicular otherwise.
Formally:
\begin{equation}
  \floss^{\text{orthogonal}} \left( \bm{P}_{jz}, \bm{D}_{jz} \right) = - \bm{D}_{jz} \bm{P}^+_{jz} + \left( \bm{D}_{jz} \bm{P}^-_{jz} \right)^2,
\end{equation}
where $\bm{P}^+_{jz} = \One\left( \bm{D}_{jz} \right) \bm{P}_{jz}$
is the matrix that keeps predictions whose concept direction is over
the threshold and cancel the other, $\bm{P}^-_{jz} = \One\left( -
\bm{D}_{jz} \right) \bm{P}_{jz}$ instead cancels prediction with
concept similarity above threshold; $\One$ is the unit step function
\begin{equation}
  \One(x) =
  \begin{cases}
    1, & x \geq 0 \\
    0, & \text{otherwise}
  \end{cases}.
\end{equation}

Also in this case, some different strategies are available. Instead of
optimizing repulsed features to be perpendicular, one could also
exploit the similarity space and force them to be inversely
correlated. Another improvement would be to normalize the score by
number of bounding box with same class label. Indeed, due to the
long-tail distribution of bounding box class labels in dataset, some
representations are moved more than other: the normalization allows to
introduce a kind of equality.

\section{Training and Inference}
\label{sec:training-and-inference}

At training stage, we are given the example $(\bm{I}, S)$ with
$\calP_{\bm{I}} = \{ bm{p}_i \}^k_{i=1}$ are bounding box proposal
built on $\bm{I}$ and $\bm{q}_j$ are queries in $S$. As noted in
Sec.~\ref{subsec:visual-branch} we use extra information from object
detector such us the probability distribution on bounding box class
labels $Pr_{Cls}(\bm{p}_i)$. Following \cite{chen2018knowledge} we fix
$k = 100$ proposals. (We studied $k$ can affect results in
Sec.~\ref{sec:data-analysis}). The parameters to be optimized include
parameters in encoding LSTM (Eq.~\ref{eq:h-star}) and projection
parameters (Eq.~\ref{eq:h-par-jz}). We use the cosine similarity as
similarity measure $\fsim$ between vectors. Our network is trained
end-to-end using Adam optimizer with gradient clipping set to $1$ and
learning rate $0.001$.

At test stage, we feed the $\bm{q}_j$ and $\calP_{\bm{I}}$ to the model. It predict a similarity score from $-1$ to $1$ for each bounding box. The grounded bounding box is calculated as
\begin{equation}
  \bm{p}_{z^*} \quad \text{s.t.} \quad z^* = \argmax_z \bm{P}_{jz}
\end{equation}