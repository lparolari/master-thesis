%!TEX root = ../dissertation.tex
% the acknowledgments section

In questo lavoro affrontiamo il problema del phrase grounding, ovvero
localizzare il contenuto di un'immagine referenziato da una frase,
usando la supervisione debole. Il phrase grounding è un problema
difficile perché richiede una conoscenza congiunta e approfondita di
entrambe le modalità testuali e visuali, ma è di rilevante importanza
in diversi ambiti di studio e applicazioni, come il visual question
answering, ovvero rispondere a domande testuali avendo a disposizioni
anche un'immagine, la ricerca di immagini o la navigazione di robot.
In questo contesto noi proponiamo un semplice modello che sfrutta la
similarità di concetto, definita come la similarità tra il concetto di
una frase e la categoria di una regione dell'immagine. Questa misura
di similarità è applicata come informazione a priori sulle predizioni
del modello ottenute in modo tradizionale. Il modello è poi
ottimizzato per massimizzare la similarità multi-modale tra immagine e
frase che la descrive, minimizzando invece la similarità multi-modale
tra immagine e una frase che invece non descrive l'immagine. Gli
esperimenti svolti mostrano risultati comparabili allo stato
dell'arte.
