%!TEX root = ../dissertation.tex
% the acknowledgments section

Non sorprendemente, la persona a cui devo di più è il mio supervisore,
Davide Rigoni, che mi ha seguito e supportato con una competenza fuori
dal comune dall'inizio di questo lavoro fino ad oggi. Mi sento
profondamente indebitato con Davide per tutto quello che mi ha
trasmesso e per il tempo che mi ha donato, aiutandomi anche su
questioni morali meno tecniche.

\vspace{0.2cm}

Ringrazio i Prof. Lamberto Ballan e Alessandro Sperduti per i preziosi
consigli che mi hanno donato e per avermi saputo guidare alla buona
riuscta di questo progetto.

\vspace{0.2cm}

Ringrazio Bryan Lucchetta e Alberto Schiabel per essere stati un punto
di riferimento in questi due anni di studio. Con loro ho condiviso
diversi pomeriggi di studio e di lavoro su vari progetti, dubbi,
preoccupazioni e successi.

\vspace{0.2cm}

Ringrazio i miei amici Davide Asticher, Michele Bedola, Mattia Porcu e
Alessio Zaina per le serate di svago (e distrazione) dai miei impegni.
Senza di loro questi due anni sarebbero stati decisamente più noiosi e
monotoni.

\vspace{0.2cm}

Un ringraziamento particolare va alla mia famiglia che mi è sempre
stata vicina e che mi continua a supportare in tutto quello che
faccio. A mia mamma che continua ad essere un punto di riferimento con
la sua generosità e gentilezza; a mio papà che è il mio esempio di
determinazione; ai miei fratelli Fabio e Davide che sono per me
modelli di felicità; alle mie nonne Fiorina ed Ottilia che hanno
sempre creduto in me; e a mio nonno Luigi che da lassù mi guarda con
il suo sorriso speciale.

\vspace{0.2cm}

Infine, un ringraziamento speciale lo devo a Giulia, la mia forza,
sorgente infinita di consigli e sostegno in ogni momento della mia
vita. Vorrei dire altro, ma cadrei presto in luoghi comuni, così come
spesso accade a chi, non essendo un poeta, si avventuri a parlare di
cose per cui le parole non bastano.

\vspace{1cm}

\begin{flushright}
  Padova, 16 dicembre 2021 \\
  -- Luca Parolari
\end{flushright}
